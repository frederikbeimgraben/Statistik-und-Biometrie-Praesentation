\newcommand{\subheader}{}

\newcommand{\SetSubheader}[1]{%
  \renewcommand{\subheader}{#1}%
}

% Renew \section to accept another argument for the subtitle; Last argument is the subheader
\let\oldsection\section

\renewcommand{\section}[2][]{%
  \SetSubheader{#1}
  \ifthenelse{\equal{#1}{}}{%
    \oldsection{#2}
  }{%
    \oldsection[#1]{#2}
  }
}

% Renew \subsection to accept another argument for the subtitle; Last argument is the subheader
\let\oldsubsection\subsection

\renewcommand{\subsection}[2][]{%
  \SetSubheader{#1}
  \ifthenelse{\equal{#1}{}}{%
    \oldsubsection{#2}
  }{%
    \oldsubsection[#1]{#2}
  }
}

\newcommand{\subsectionwithpage}[2][]{
  \SetSubheader{#1}
  \ifthenelse{\equal{#1}{}}{%
    \oldsubsection{#2}
  }{%
    \oldsubsection[#1]{#2}
  }
  \begin{frame}
    \subsectionpage
  \end{frame}
}

\AtBeginSection[]
{
  \begin{frame}
    \sectionpage
  \end{frame}
}

\newcommand{\ExternalLinkSymbol}{%
    \tikz[x=1.2ex, y=1.2ex, baseline=-0.05ex]{% 
        \begin{scope}[x=1ex, y=1ex]
            \clip (-0.1,-0.1) 
                --++ (-0, 1.2) 
                --++ (0.6, 0) 
                --++ (0, -0.6) 
                --++ (0.6, 0) 
                --++ (0, -1);
            \path[draw, 
                line width = 0.5, 
                rounded corners=0.5] 
                (0,0) rectangle (1,1);
        \end{scope}
        \path[draw, line width = 0.5] (0.5, 0.5) 
            -- (1, 1);
        \path[draw, line width = 0.5] (0.6, 1) 
            -- (1, 1) -- (1, 0.6);
        }
    }

\newcommand{\ExternalLink}[2]{
    \href{#1}{#2 \ExternalLinkSymbol}
}

\SetTranslationKey{\published}{Veröffentlicht}{Published}

\newcommand{\Literature}[7]{
    % Parameters:
    % - citekey     : Required
    % - Authors     : Required
    % - Title       : Required
    % - Year        : Required
    % - Journal     : Optional
    % - DOI         : Optional
    % - Note        : Optional

    % Local variables
    \def\citekey{#1}
    \def\authors{#2}
    \def\title{#3}
    \def\year{#4}
    \def\journal{#5}
    \def\doi{#6}
    \def\note{#7}

    % Actual output
    \bibitem{citekey}
    \blenderfont
    \textbf{\authors}\\
    \emph{\title}\\
    \emph{\published}~\year~\ifthenelse{\equal{\journal}{}}{}{\emph{in}~\journal}\\
    % Check if DOI is a full URL - if so omit the DOI prefix
    \ifthenelse{\equal{\doi}{}}{}{%
        DOI:~\ExternalLink{http://dx.doi.org/\doi}{\doi}
    }
    \note
}

% Command to include image with a caption
\newcommand{\ImagePage}[6][0.5\paperheight]{
    % Parameters:
    % - Path        : Required
    % - Caption     : Required
    % - Source      : Optional

    % Local variables
    \def\imageheight{#1}
    \def\labelid{#2}
    \def\path{#3}
    \def\captiontext{#4}
    \def\sourceURL{#5}
    \def\sourceCaption{#6}

    % Actual output
    \begin{frame}
        \frametitle{\captiontext}
        \begin{figure}
            \includegraphics[height=\imageheight]{\path}
            \caption{
                \captiontext\\
                \ifthenelse{\equal{\sourceURL}{}}{}{\ExternalLink{\sourceURL}{\sourceCaption}}
            }
            \label{fig:\labelid}
        \end{figure}
    \end{frame}
}

\def\shadowshift{3pt,-3pt}
\def\shadowradius{6pt}

\colorlet{innercolor}{black!60}
\colorlet{outercolor}{gray!05}

% this draws a shadow under a rectangle node
\newcommand\drawshadow[1]{
    \begin{pgfonlayer}{shadow}
        \shade[outercolor,inner color=innercolor,outer color=outercolor] ($(#1.south west)+(\shadowshift)+(\shadowradius/2,\shadowradius/2)$) circle (\shadowradius);
        \shade[outercolor,inner color=innercolor,outer color=outercolor] ($(#1.north west)+(\shadowshift)+(\shadowradius/2,-\shadowradius/2)$) circle (\shadowradius);
        \shade[outercolor,inner color=innercolor,outer color=outercolor] ($(#1.south east)+(\shadowshift)+(-\shadowradius/2,\shadowradius/2)$) circle (\shadowradius);
        \shade[outercolor,inner color=innercolor,outer color=outercolor] ($(#1.north east)+(\shadowshift)+(-\shadowradius/2,-\shadowradius/2)$) circle (\shadowradius);
        \shade[top color=innercolor,bottom color=outercolor] ($(#1.south west)+(\shadowshift)+(\shadowradius/2,-\shadowradius/2)$) rectangle ($(#1.south east)+(\shadowshift)+(-\shadowradius/2,\shadowradius/2)$);
        \shade[left color=innercolor,right color=outercolor] ($(#1.south east)+(\shadowshift)+(-\shadowradius/2,\shadowradius/2)$) rectangle ($(#1.north east)+(\shadowshift)+(\shadowradius/2,-\shadowradius/2)$);
        \shade[bottom color=innercolor,top color=outercolor] ($(#1.north west)+(\shadowshift)+(\shadowradius/2,-\shadowradius/2)$) rectangle ($(#1.north east)+(\shadowshift)+(-\shadowradius/2,\shadowradius/2)$);
        \shade[outercolor,right color=innercolor,left color=outercolor] ($(#1.south west)+(\shadowshift)+(-\shadowradius/2,\shadowradius/2)$) rectangle ($(#1.north west)+(\shadowshift)+(\shadowradius/2,-\shadowradius/2)$);
        \filldraw ($(#1.south west)+(\shadowshift)+(\shadowradius/2,\shadowradius/2)$) rectangle ($(#1.north east)+(\shadowshift)-(\shadowradius/2,\shadowradius/2)$);
    \end{pgfonlayer}
}

% create a shadow layer, so that we don't need to worry about overdrawing other things
\pgfdeclarelayer{shadow} 
\pgfsetlayers{shadow,main}


\newcommand\shadowimage[2][]{%
\begin{tikzpicture}
\node[anchor=south west,inner sep=0] (image) at (0,0) {\includegraphics[#1]{#2}};
\drawshadow{image}
\end{tikzpicture}}

\newcommand\shadowsvg[2][]{%
\begin{tikzpicture}
\node[anchor=south west,inner sep=0] (image) at (0,0) {\includesvg[#1]{#2}};
\drawshadow{image}
\end{tikzpicture}}